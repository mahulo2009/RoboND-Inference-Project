%% V1.0
%% by Manuel Huertas, manuel.huertas.lopez@gmail.com
%% 

%% 

\documentclass[10pt,journal,compsoc]{IEEEtran}

\usepackage[pdftex]{graphicx}    
\usepackage{cite}
\hyphenation{op-tical net-works semi-conduc-tor}


\begin{document}
	
	\title{Coin Classification}
	
	\author{Manuel Huertas L\'opez}
	
	\markboth{Inference project, Robotic Nanodegree, Udacity}%
	{}
	\IEEEtitleabstractindextext{%
		
		\begin{abstract}
			
			A abstract is meant to be a summary of all of the relevant points in your presented work. It is designed to present a high-level overview of the report, providing just enough detail to convey the necessary information The abstract may often mention a one-sentence summary of the results.  While the type of voice chosen for the paper (active or passive) may be up for debate, you should avoid the use of “I” and “me” in the report. It usually is kept to a length of 150 - 200 words. 
			Example: You should not write, “I present two different neural networks for classifying my data”. Instead, you should try to say, “Two different neural networks are used for classification”.
		\end{abstract}
		
		% Note that keywords are not normally used for peerreview papers.
		\begin{IEEEkeywords}
			Robot, IEEEtran, Udacity, \LaTeX, deep learning.
	\end{IEEEkeywords}}
	
	
	\maketitle
	\IEEEdisplaynontitleabstractindextext
	\IEEEpeerreviewmaketitle
	\section{Introduction}
	\label{sec:introduction}
	
	\IEEEPARstart{T}{he} introduction should provide some material regarding the history of the problem, why it is important and what is intended to be achieved. If there exists any previous attempts to solve this problem, this is a great place to note these while conveying the differences in your approach (if any). The intent is to provide enough information for the reader to understand why this problem is interesting and setting up the conversation for the solution you have provided
	Use this space to introduce your robotic inference idea and how you wish to apply it. 
	If you have any papers / sites you have referenced for your idea, please make sure to cite them.
	
	%example for inserting image
	\begin{figure}[thpb]
		\centering
		\includegraphics[width=\linewidth]{RobotRevolution5}
		\caption{Robot Revolution.}
		\label{fig:robot1}
	\end{figure}
	
	\subsection{Subsection Heading Here}
	Subsection text here.
	
	\subsubsection{Subsubsection Heading Here}
	Subsubsection text here.
	
	
	\begin{table}[h]
		\caption{Table}
		\label{table_example}
		\begin{center}
			\begin{tabular}{|c||c|}
				\hline
				One & Two\\
				\hline
				Three & Four\\
				\hline
			\end{tabular}
		\end{center}
	\end{table}
	
	
	
	
	
	\section{Background / Formulation}
	At this stage, you should begin diving into the technical details of your approach by explaining to the reader how parameters were defined, what type of network was chosen, and the reasons these items were performed. This should be factual and authoritative, meaning you should not use language such as “I think this will work” or “Maybe a network with this architecture is better..”. Instead, focus on items similar to, ”A 3-layer network architecture was chosen with X, Y, and Z parameters” 
	Explain why you chose the network you did for the supplied data set and then why you chose the network used for your robotic inference project. \cite{lamport1994latex}
	
	%example for Bullet point list
	
	\begin{itemize}
		\item example
		\end {itemize}
		
		
		
		%example for numbered list
		\begin{enumerate}
			\item example
			
		\end{enumerate}
		
		\section{Data Acquisition}
		This section should discuss the data set. Items to include are the number of images, size of the images, the types of images (RGB, Grayscale, etc.), how these images were collected (including the method). Providing this information is critical if anyone would like to replicate your results. After all, the intent of reports such as these are to convey information and build upon ideas so you want to ensure others can validate your process.
		Justifying why you gathered data in this way is a helpful point, but sometimes this may be omitted here if the problem has been stated clearly in the introduction.
		It is a great idea here to have at least one or two images showing what your data looks like for the reader to visualize.
		
		\section{Results}
		This is typically the hardest part of the report for many. You want to convey your results in an unbiased fashion. If you results are good, you can objectively note this. Similarly, you may do this if they are bad as well. You do not want to justify your results here with discussion; this is a topic for the next session. 
		Present the results of your robotics project model and the model you used for the supplied data with the appropriate accuracy and inference time
		For demonstrating your results, it is incredibly useful to have some charts, tables, and/or graphs for the reader to review. This makes ingesting the information quicker and easier.
		
		\section{Discussion}
		This is the only section of the report where you may include your opinion. However, make sure your opinion is based on facts. If your results are poor, make mention of what may be the underlying issues. If the results are good, why do you think this is the case? Again, avoid writing in the first person (i.e. Do not use words like “I” or “me”). If you really find yourself struggling to avoid the word “I” or “me”; sometimes, this can be avoid with the use of the word “one”. As an example: instead of : “I think the accuracy on my dataset is low because the images are too small to show the necessary detail” try: “one may believe the accuracy on the dataset is low because the images are too small to show the necessary detail”. They say the same thing, but the second avoids the first person. 
		Reflect on which is more important, inference time or accuracy, in regards to your robotic inference project.
		
		\section{Conclusion / Future work}
		This section is intended to summarize your report. Your summary should include a recap of the results, did this project achieve what you attempted, and is this a commercially viable product? 
		For Future work,address areas of work that you may not have addressed in your report as possible next steps. For future work, this could be due to time constraints, lack of currently developed methods / technology, and areas of application outside of your current implementation. Again, avoid the use of the first-person.
		
		\bibliography{bib}
		\bibliographystyle{ieeetr}
		
	\end{document}
	
	
	Start with our Templates
	Overleaf is perfect for all types of projects — from papers and presentations to newsletters, CVs and much more! It's also a great way to learn how to use LaTeX and produce professional looking projects quickly.
	
	Make your Own
	Upload or create templates for journals you submit to and theses and presentation templates for your institution. Just create it as a project on Overleaf and use the publish menu. It's free! No sign-up required.
	
	Follow us for More
	New template are added all the time. Follow us on twitter for the highlights!
	
	Overleaf is a free online collaborative LaTeX editor. No sign up required.
	Learn more
	Search for more Templates, Articles and Examples
	Search
	Search…
	SEARCH
	Latest News
	February 16, 2018
	Tip of the Week: Speed up compile time
	February 15, 2018
	Slides and videos from the recent celebration of Don Knuth’s 80th birthday
	Overleaf
	Getting Started
	Templates & Gallery
	iPad & Tablet
	Plans & Pricing
	Referrals
	Teams
	Support
	Help & FAQs
	Privacy & Terms
	Security Overview
	Developers
	Company
	About us
	Contact us
	Partners
	Advisors
	Press
	Blog
	Jobs
	
	Writelatex Limited, 3rd Floor, 207 Regent Street, London, W1B 3HH, UKCopyright © 2018. All rights reserved.

