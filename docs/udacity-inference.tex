\documentclass[10pt,journal,compsoc]{IEEEtran}

\usepackage[pdftex]{graphicx}    
\usepackage{cite}
\hyphenation{op-tical net-works semi-conduc-tor}


\begin{document}
	
\title{Robotics Software Engineer Nanodegree: Inference Project}
\author{Manuel Huertas L\'opez}

\markboth{Inference project, Udacity}{}
\IEEEtitleabstractindextext{%
	
\begin{abstract}
This project contains two parts. In the first part of the project a data set is provided and a neural network must be trained in order to achieve an inference time less than 10 ms and accuracy greater than 75\%. In the second part of the project an original idea for a robotic inference system must be selected. Then the data must be collected and a network trained. The project selected, identifying coins and classifying them, has been taking in account that in a robotic system we want to perceive the world, make a decision based on that perception and them act upon this decision. The neural network can be used to automate the process of counting  coins.
\end{abstract}


\begin{IEEEkeywords}
inference project, udacity, deep learning.
\end{IEEEkeywords}}
	
	
\maketitle
\IEEEdisplaynontitleabstractindextext
\IEEEpeerreviewmaketitle
\section{Introduction}
\label{sec:introduction}
	
	
\IEEEPARstart{T}{he} field of robotics has evolved during the last decades. During the industrial revolution, the robots that were manufactured could operate repetitively with high precision. The current robots can move around the scene detecting automatically obstacles and reacting in real time to changes in the environment. The flexibility needed for the new era of robots can be beneficed of the neural networks applied to image classification or semantic segmentation. By using this techniques a trained robot can rapidly identify objects in the scene and react accordantly.  It is not only provided with sensors to avoid obstacles, it can detect object in the scene, classified them, and to make decision bases on this classifications.

A lot of effort to provide the robots these capabilities came from collecting data to train the neural network and to define the network architecture itself. In the present work an example, classify coins, have been selected in order to evaluate the data collection process and the architecture selection an final verification.

	
\section{Background / Formulation}


At this stage, you should begin diving into the technical details of your approach by explaining to the reader how parameters were defined, what type of network was chosen, and the reasons these items were performed. This should be factual and authoritative, meaning you should not use language such as “I think this will work” or “Maybe a network with this architecture is better..”. Instead, focus on items similar to, ”A 3-layer network architecture was chosen with X, Y, and Z parameters” 
Explain why you chose the network you did for the supplied data set and then why you chose the network used for your robotic inference project. \cite{lamport1994latex}
		
\section{Data Acquisition}
This section should discuss the data set. Items to include are the number of images, size of the images, the types of images (RGB, Grayscale, etc.), how these images were collected (including the method). Providing this information is critical if anyone would like to replicate your results. After all, the intent of reports such as these are to convey information and build upon ideas so you want to ensure others can validate your process.
Justifying why you gathered data in this way is a helpful point, but sometimes this may be omitted here if the problem has been stated clearly in the introduction.
It is a great idea here to have at least one or two images showing what your data looks like for the reader to visualize.

\section{Results}
This is typically the hardest part of the report for many. You want to convey your results in an unbiased fashion. If you results are good, you can objectively note this. Similarly, you may do this if they are bad as well. You do not want to justify your results here with discussion; this is a topic for the next session. 
Present the results of your robotics project model and the model you used for the supplied data with the appropriate accuracy and inference time
For demonstrating your results, it is incredibly useful to have some charts, tables, and/or graphs for the reader to review. This makes ingesting the information quicker and easier.

\section{Discussion}
This is the only section of the report where you may include your opinion. However, make sure your opinion is based on facts. If your results are poor, make mention of what may be the underlying issues. If the results are good, why do you think this is the case? Again, avoid writing in the first person (i.e. Do not use words like “I” or “me”). If you really find yourself struggling to avoid the word “I” or “me”; sometimes, this can be avoid with the use of the word “one”. As an example: instead of : “I think the accuracy on my dataset is low because the images are too small to show the necessary detail” try: “one may believe the accuracy on the dataset is low because the images are too small to show the necessary detail”. They say the same thing, but the second avoids the first person. 
Reflect on which is more important, inference time or accuracy, in regards to your robotic inference project.
		
\section{Conclusion / Future work}
This section is intended to summarize your report. Your summary should include a recap of the results, did this project achieve what you attempted, and is this a commercially viable product? 
For Future work,address areas of work that you may not have addressed in your report as possible next steps. For future work, this could be due to time constraints, lack of currently developed methods / technology, and areas of application outside of your current implementation. Again, avoid the use of the first-person.
		
\bibliography{bib}
\bibliographystyle{ieeetr}
		
\end{document}
	
	
